\documentclass[12pt]{article}
\usepackage{graphicx}
\usepackage{hyperref}
\usepackage{fancyvrb}


\title{Getting Started With Glimpse}
\author{
        Metron Scientific Solutions\\
}
\date{\today}

\begin{document}
\maketitle

\begin{center}
\includegraphics[width=.4\textwidth]{images/MetronLogo.png}
\end{center}

\begin{center}
\includegraphics[width=.4\textwidth]{images/GlimpseLogo.png}
\end{center}

\section{What Is Glimpse?}

Glimpse is a Java library for creating interactive data visualization applications which integrates into Swing-based Java applications. Glimpse simplifies utilization of powerful GPU hardware like programmable shader pipelines and texture memory to display large quantities of data at interactive speeds. Complicated plots and visualizations can be created by arranging pre-built components using the popular MigLayout manager. Axes, timelines, color scales, cursor crosshairs, and other plot embellishments are provided out of the box, as well as common plot arrangements and layers for visualizing common data types (like histograms, line charts, heat maps, and tree maps). Glimpse is also excellent for constructing geographic visualization applications. Layers to display bathymetry and topographic contours, electronic navigation charts, land contours, and arbitrary projected images are built in.

\section{Requirements}

Glimpse requires Java 1.6.0. Other software dependencies are included with binary distributions or available using Maven (\url{http://maven.apache.org/}). A wide range of modern graphics cards with OpenGL 2.0 compatibility will work with Glimpse.

\section{Getting Started}

The best way to get familiar with what Glimpse offers and how it works is to check out the large library of examples in the \emph{core-examples} and \emph{extras-examples} modules. Each example consists of a single runnable java class that demonstrates a feature of Glimpse. Figure~\ref{simpleplot} shows the output of the HeatMapExample.

\begin{figure}
  \centering
    \includegraphics[width=0.5\textwidth]{images/Heatmap_empty.png}
  \caption{Simple Glimpse Plot}
\label{simpleplot}
\end{figure}

\section{Glimpse Canvas}

All Glimpse applications start with a GlimpseCanvas which represents something onto which OpenGL rendering can take place. For Swing applications, this is a SwingGlimpseCanvas, which is also a JPanel and can be directly incorporated into an existing Swing application. Other GlimpseCanvas implementations exist for drawing to off-screen buffers.

The following code snippit is all that is necessary to bring up a Swing JFrame containing a GlimpseCanvas:

\begin{samepage}
\begin{verbatim}

    public static void main( String[] args )
    {
        Jogular.initJogl( );
        SwingGlimpseCanvas canvas = new SwingGlimpseCanvas( );
        RepaintManager.newRepaintManager( canvas );
        JFrame frame = new JFrame( "Glimpse Example" );
        frame.add( canvas );
        frame.setSize( 100, 100 );
        frame.setVisible( true );
    }

\end{verbatim}
\end{samepage}

The line \emph{Jogular.initJogl( )} automatically places the necessary OpenGL native libraries on the \emph{java.library.path}. Then a new \emph{SwingGlimpseCanvas} is constructed. Other constructor arguments allow advanced features like sharing GLContexts between canvases, but the simple zero argument constructor suffices here. The call to \emph{newRepaintManager( )} adds the newly constructed canvas to a threaded manager which will repaint the canvas as appropriate. A Glimpse application should have only one instance of RepaintManager which all canvases are registered with. Finally, the canvas can be added to the Swing frame like any other Swing component.

When this snippet is run, a window with a 100 pixel by 100 pixel black square will appear on the screen. In order to get something more interesting showing up, we need to add GlimpseLayouts and GlimpsePainters to the canvas.

\section{Glimpse Layout and Glimpse Painter}

Glimpse provides tools to break up a single GlimpseCanvas into many (possibly nested) logical plotting areas which can each be painted on and receive mouse events. Each plotting area is defined by a GlimpseLayout and arranged using Mig Layout (\url{http://www.miglayout.com/}). Figure~\ref{simpleplotoutlines} outlines the GlimpseLayouts from HeatMapExample. Although everything is rendered in a single OpenGL canvas, Glimpse allows mouse listeners to be attached to any GlimpseLayout and allows painting to be performed inside a GlimpseLayout without bleeding into adjacent layouts.

\begin{figure}
  \centering
    \includegraphics[width=0.5\textwidth]{images/Heatmap_red_layout_bounds.png}
  \caption{Glimpse Plot with Glimpse Layouts outlined}
\label{simpleplotoutlines}
\end{figure}

Even better, GlimpseLayouts can be nested, allowing easy construction of very complicated plots. Figure~\ref{multiplotoutlines} demonstrates nesting of three simple heat map plots into a single hybrid plot. The GlimpseLayouts are again outlined in red. The example class SimpleLayoutExample demonstrates how Mig Layout is used to achieve this arrangement.

\begin{figure}
  \centering
    \includegraphics[width=0.5\textwidth]{images/Multi_Heatmap_red_layout.png}
  \caption{Nested Plots with Glimpse Layouts outlined}
\label{multiplotoutlines}
\end{figure}

Once the plotting areas have been defined, OpenGL rendering inside GlimpseLayouts is performed by GlimpsePainters. Multiple GlimpsePainters can be added to a GlimpseLayout and act like layers. Glimpse provides lots of pre-built painter in the \emph{com.metsci.glimpse.painter} package. However, most applications also want to perform custom rendering. The \emph{painter.base} package contains the GlimpsePainter interface, as well as a number of abstract helper implementations. In most cases, GlimpseDataPainter2D is a good place to start. It handles setting up the OpenGL viewport and scissor regions and projection matrix so that \emph{glVertex()} calls will draw to the correct screen location based on the GlimpseLayout being painted into and the bounds of the data axes.

The following example demonstrates a very simple GlimpsePainter which draws a blue diagonal line from \((5.0,5.0)\) to \((10.0,10.0)\) in data space. When axes are drawn onto the same GlimpseLayout, the line will be at those coordinates, regardless of how the plot is panned or zoomed (see Figure \ref{simplepainter}).

\begin{samepage}
\begin{verbatim}

public class SimplePainter extends GlimpseDataPainter2D
{
    @Override
    public void paintTo( GL gl, GlimpseBounds bounds, Axis2D axis )
    {
        gl.glColor3f( 0.0f, 0.0f, 1.0f );
        gl.glLineWidth( 3.0f );
        
        gl.glBegin( GL.GL_LINES );
        try
        {
            gl.glVertex2d( 5.0, 5.0 );
            gl.glVertex2d( 10.0, 10.0 );
        }
        finally
        {
            gl.glEnd( );
        }
    }
}

\end{verbatim}
\end{samepage}

\begin{figure}
  \centering
    \includegraphics[width=0.5\textwidth]{images/simple_painter.png}
  \caption{Simple Glimpse Painter}
\label{simplepainter}
\end{figure}

\end{document}
